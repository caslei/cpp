%!TEX root=paper/paper.tex
\chapter{Basic C Language}\label{sec:basic_chapter}

\section{Keywords}
由ANSI标准定义的C语言关键字共32个 :

\begin{table}
\begin{tabular}{|l|l|l|l|l|l|l|}
\hline
auto      & break  & case      & char    & const  	   & continue	  & default   \\
\hline
do        & double &  else     & enum    & extern      & float        & for       \\
\hline
goto      & if     & int       & long    & register    &  return      & short     \\
\hline
signed    & sizeof & static    & sturct  & switch      & typedef      & union     \\
\hline
unsigned  & void   & volatile  &  while  & \textit{inline}    & \textit{restrict}     & \textit{_Bool_Complex}   \\
\hline
\textit{_Imaginary} & \textit{_Alignas} & \textit{_Alignof}  & \textit{_Atomic}   & \textit{_Static_assert}  &  \textit{_Noreturn}  & \textit{_Thread_local}     \\
\hline
\textit{_Generic}  & & & & & &  \\
\hline
\end{tabular}
\end{table}

根据关键字的作用,可以将关键字分为\textit{数据类型关键字}和\textit{流程控制关键字}两大类。

\begin{enumerate}[label=\arabic*)]
	\item \textit{数据类型关键字}
	\begin{enumerate}
		\item 基本数据类型(5个)
		\begin{enumerate}
			\item void: 声明函数无返回值或无参数,声明无类型指针,显式丢弃运算结果
			\item char: 字符型类型数据,属于整型数据的一种
			\item int: 整型数据,通常为编译器指定的机器字长
			\item float: 单精度浮点型数据
			\item double: 双精度浮点型数据
		\end{enumerate}
		\item 类型修饰关键字(4个)
		\begin{enumerate}
			\item short: 修饰int,短整型数据,可省略被修饰的int。
			\item long: 修饰int,长整形数据,可省略被修饰的int
			\item signed: 修饰整型数据,有符号数据类型
			\item unsigned: 修饰整型数据,无符号数据类型
		\end{enumerate}
		\item 复杂类型关键字(5个)
		\begin{enumerate}
			\item struct: 结构体声明
			\item union: 共用体声明
			\item enum: 枚举声明
			\item typedef: 声明类型别名
			\item sizeof: 得到特定类型或特定类型变量的字节大小
		\end{enumerate}
		\item 存储级别关键字(6个)
		\begin{enumerate}
			\item auto: 指定为自动变量,由编译器自动分配及释放。通常在栈上分配
			\item static: 指定为静态变量,分配在静态变量区,修饰函数时,指定函数作用域为文件内部
			\item register: 指定为寄存器变量,建议编译器将变量存储到寄存器中使用,也可以修饰函数形参,建议编译器通过寄存器而不是堆栈传递参数
			\item extern: 指定对应变量为外部变量,即在另外的目标文件中定义,可以认为是约定由另外文件声明的对象的一个``引用"
			\item const: 与volatile合称“cv特性”,指定变量不可被当前线程/进程改变(但有可能被系统或其他线程/进程改变)
			\item volatile: 与const合称“cv特性”,指定变量的值有可能会被系统或其他进程/线程改变,强制编译器每次从内存中取得该变量的值
		\end{enumerate}	
	\end{enumerate}
	\item \textit{流程控制关键字}
	\begin{enumerate}
		\item 跳转结构(4个)
		\begin{enumerate}
			\item return :用在函数体中,返回特定值(或者是void值,即不返回值)
			\item continue :结束当前循环,开始下一轮循环
			\item break :跳出当前循环或switch结构
			\item goto :无条件跳转语句
		\end{enumerate}
		\item 分支结构(5个)
		\begin{enumerate}
			\item if :条件语句
			\item else :条件语句否定分支(与if连用)
			\item switch :开关语句(多重分支语句)
			\item case :开关语句中的分支标记
			\item default :开关语句中的``其他分治,可选。
		\end{enumerate}
		\item 循环结构(3个)
		\begin{enumerate}
			\item for :for循环结构
			\item do :do循环结构
			\item while :while循环结构
		\end{enumerate}	
	\end{enumerate}
\end{enumerate}


