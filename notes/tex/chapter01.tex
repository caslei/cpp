%!TEX root=paper/paper.tex
\chapter{Basic C Language}\label{sec:basic_chapter}


\section{Keywords}

% \begin{enumerate}
% \item{简洁紧凑、灵活方便}: ANSI C一共只有32个关键字,9种控制语句,程序书写形式自由,{\color{red} 区分大小写}。它把高级语言的基本结构和语句与低级语言的实用性结合起来,可以像汇编语言一样对位、字节和地址进行操作,而这三者是计算机最基本的工作单元。

% \item{运算符丰富}: C语言的运算符包含的范围很广泛,共有34种运算符,他把括号、赋值、强制类型转换等都作为运算符处理,从而使C语言的运算类型极其丰富,表达式类型多样化。灵活使用各种运算符可以实现在其它高级语言中难以实现的运算。

% \item{数据类型多样}: 它包含的数据类型有:整型、浮点型、字符型、数组类型、指针类型、结构体类型、共用体类型等,能实现各种复杂数据结构的运算。指针概念的引入使程序效率更高,且计算功能、逻辑判断功能强大,同时对于不同的编译器也有各种强大的扩展功能。

% \item{另外}: C语言如此丰富数据类型及强大指针功能,其对硬件的管控能力极强,所以许多操作系统内核及MCU芯片程序开发都偏爱硬件。	
% \begin{enumerate} 


由ANSI标准定义的C语言关键字共32个 :
\begin{table}[htbp]
\centering
\begin{tabular}{|l|l|l|l|l|l|l|}
\hline
auto      & break  & case      & char    & const  	   & continue	  & default   \\
%\hline
do        & double &  else     & enum    & extern      & float        & for       \\
%\hline
goto      & if     & int       & long    & register    &  return      & short     \\
%\hline
signed    & sizeof & static    & sturct  & switch      & typedef      & union     \\
%\hline
unsigned  & void   & volatile  &  while  & \emph{inline}    & \emph{restrict}     & \emph{\_Bool\_Complex}   \\
%\hline
\emph{\_Imaginary} & \emph{\_Alignas} & \emph{\_Alignof}  & \emph{\_Atomic}   & \emph{\_Static\_assert}  &  \emph{\_Noreturn}  & \emph{\_Thread\_local}     \\
%\hline
\emph{\_Generic}  & & & & & &  \\
\hline
\end{tabular}
\end{table}





根据关键字的作用,可以将关键字分为\textbf{数据类型关键字}和\textbf{流程控制关键字}两大类。
\begin{table}[htbp]
\begin{center}
\small{\begin{tabular}{|p{0.075\linewidth}|p{0.075\linewidth}|p{0.8\linewidth}|}
\hline  
\multicolumn{3}{|l|}{\textbf{数据类型关键字}} \\ \hline
\multirow{5}{*}{\begin{tabular}{c} 基本类 \\型关键 \\字(5) \\\end{tabular}}  & void & 声明函数无返回值或无参数,声明无类型指针,显式丢弃运算结果 \\
																			& char & 字符型类型数据,属于整型数据的一种 \\  
																			& int & 整型数据,通常为编译器指定的机器字长 \\ 
																			& float & 单精度浮点型数据 \\  
																			& double & 双精度浮点型数据 \\ \hline
\multirow{4}{*}{\begin{tabular}{c} 类型修 \\饰关键 \\字(4) \\\end{tabular}}  &  short & 修饰int,短整型数据,可省略被修饰的int \\
																			& long & 修饰int,长整形数据,可省略被修饰的int  \\ 
																			& signed & 修饰整型数据,有符号数据类型 \\ 
																			& unsigned & 修饰整型数据,无符号数据类型 \\ \hline
\multirow{5}{*}{\begin{tabular}{c} 复杂类 \\型关键 \\字(5) \\\end{tabular}}  &  struct & 结构体声明 \\
																			& union & 共用体声明  \\ 
																			& enum & 枚举声明 \\ 
																			& typedef & 声明类型别名 \\ 
																			& sizeof & 计算数据类型或变量长度(即所占字节数) \\ \hline
\multirow{6}{*}{\begin{tabular}{c} \\~\\~\\ 存储级 \\别关键 \\字(6) \\\end{tabular}}  & auto & 指定为自动变量,由编译器自动分配及释放。通常在栈上分配 \\ 
																			& static & 指定为静态变量,分配在静态变量区,修饰函数时,指定函数作用域为文件内部 \\  
																			& register & 指定为寄存器变量,建议编译器将变量存储到寄存器中使用,也可以修饰函数形参,建议编译器通过寄存器而不是堆栈传递参数 \\  
																			& extern & 指定对应变量为外部变量,即在另外的目标文件中定义,可以认为是约定由另外文件声明的对象的一个``引用"\\  
																			& const & (1) 声明变量为常量, (2) char *const p; 到char的const指针 (指针本身为const), (3) char const *p; 到const char的指针 (所指对象为const)\\ 
																			& volatile & 指定变量的值有可能会被系统或其他进程/线程改变,强制从内存中取得变量的值, (2) 用volatile定义的变量会在程序外被改变,每次都必须从内存中读取,而不能重复使用放在cache或寄存器中的备份, (3) volatile关键词影响编译器编译的结果,用volatile声明的变量表示该变量随时可能发生变化,与该变量有关的运算,不要进行编译优化,以免出错 \\  \hline
\multicolumn{3}{|l|}{\textbf{流程控制关键字}}  \\  \hline
\multirow{4}{*}{\begin{tabular}{c} 跳转结 \\构关键 \\字(4) \\\end{tabular}}  & return & 用在函数体中,返回特定值(可以带参数,也可不带参数) \\
																			& continue & (1) 跳过本次循环余下的语句,开始下一轮循环, (2) 只作用于离它最近的循环:for, while, do...while \\ 
																			& break & (1) 终止离它最近的循环, (2) 跳出switch结构, \\
																			& goto & 无条件跳转语句 \\ \hline
\multirow{5}{*}{\begin{tabular}{c} 分支结 \\构关键 \\字(5) \\\end{tabular}}  & if & 条件语句 \\
																			& else & 条件语句否定分支(与if连用) \\ 
																			& switch & 开关语句(多重分支语句) \\  
																			& case & 开关语句中的分支标记 \\  
																			& default & 开关语句中的``其他"分支,可选。 \\ \hline
\multirow{3}{*}{\begin{tabular}{c} 循环结 \\构关键 \\字(3) \\\end{tabular}}  & for & for循环结构,for(1;2;3)4;的执行顺序为1->2->4->3->2...循环,其中2为循环条件 \\ 
																			& do & do循环结构,do 1; while(2); 的执行顺序是 1->2->1...循环,2为循环条件 \\ 
																			& while & while循环结构,while(1) 2; 的执行顺序是1->2->1...循环,1为循环条件  \\  \hline
\end{tabular} }
\end{center}
\end{table}

% \begin{tikzpicture}
% \node [mybox](box){
%     \begin{minipage}{0.3\textwidth}
%         \url{https://www.cnblogs.com/chio/archive/2008/06/18/1225028.html}:\\
%         \textit{void assert(int expression);}
%         \begin{center}
%            \small{\begin{tabular}{lp{4.5cm} l}
%                 \textit{1:} & 若\textit{expression==0}, 输出错误提示,终止程序 \\ \hline
%                 \textit{2:} & 若\textit{expression!=0}, 继续执行下条语句 \\ \hline
%                 \textit{3:} & 若在\textit{assert.h}前定义\textit{\#define NDEBUG} 则所有 \textit{assert}失效
%             \end{tabular} }
%         \end{center}
%     \end{minipage}
% }; \\
% \node[fancytitle, right=10pt] at (box.north west) {<assert.h>};
% \end{tikzpicture