%!TEX root=paper/paper.tex
\subsection{Learning the classifier}\label{sec:clf_classifier}

\PM{Classifiers}
We have so far assumed that $g$ can combine an arbitrary subset of features and provide a distribution $P(Y = y)$---for example, a Gaussian Naive Bayes (NB) model trained on the fully-observed data.
However, a Naive Bayes classifier suffers from its restrictive independence assumptions.
Since discriminative classifiers commonly provide better performance, we'd like to use a \textbf{logistic regression} classifier.
Nearest Neighbor methods also provide a robust classifier for partially observed data, but are slow for large datasets.
We consider them in this exposition and in preliminary experiments reported in \autoref{sec:imputation_evaluation}, but do not use them in final policy learning experiments due to this problem.

\PM{Missing features}
Note that at test time, some feature values are missing.
If the classifier is trained exclusively on fully-observed data, then the feature value statistics at test time will not match, resulting in poor performance.
Therefore, we need to learn classifier weights on a distribution of data that exhibits the pattern of missing features induces by the policy $\pi$, and/or try to intelligently impute unobserved values.

\begin{algorithm}[]
\SetKwFunction{ComputeRewards}{ComputeRewards}
\SetKwFunction{GatherSamples}{GatherSamples}
\SetKwFunction{UpdatePolicy}{UpdatePolicy}
\SetKwFunction{UpdateClassifier}{UpdateClassifier}

\SetAlgoLined
\KwIn{$\mathcal{D} = \{x_n, y_n\}_{n=1}^N$; $\mathcal{L}_\mathcal{B}$}
\KwResult{Trained $\pi$, $g$}
\BlankLine
$\pi_0 \leftarrow$ random\;
\For {$i \leftarrow 1$ \KwTo max\_iterations} {
    States, Actions, Costs, Labels $\leftarrow$ \GatherSamples{$\mathcal{D}$, $\pi_{i-1}$}\;
    $g_i \leftarrow$ \UpdateClassifier{States, Labels}\;
    Rewards $\leftarrow$ \ComputeRewards{States, Costs, Labels, $g_i, \mathcal{L}_\mathcal{B}, \gamma$}\;
    $\pi_i \leftarrow$ \UpdatePolicy{States, Actions, Rewards}\;
}
\caption{Because reward computation depends on the classifier, and the distribution of states depends on the policy, $g$ and $\pi$ are trained iteratively.\label{alg:learning}}
\end{algorithm}


\PM{Joint learning}
At the same time, learning the policy depends on the classifier $g$, used in the computation of the rewards.
For this reason, the policy and classifier need to be learned jointly: \autoref{alg:learning} gives the iterative procedure.
In summary, we start from random $\pi$ and $g$, gather a batch of trajectories.
The batch is used to update both $g$ and $\pi$.
Then new trajectories are generated with the updated $\pi$, rewards are computed using the updated $g$, and the process is repeated.

\subsubsection{Unobserved value imputation}

\PM{Formulation}
Unlike the Naive Bayes classifier, the logistic regression classifier is not able to use an arbitrary subset of features $\mathcal{H}_\pi$, but instead operates on feature vectors of a fixed size.
To represent the feature vector of a fully observed instance, we write $\mathbf{x} = [h_1(x), \dots, h_f(x)]$.
In case that $\mathcal{H}_\pi \subset \mathcal{H}$, we need to fill in unobserved feature values in the vector.

\PM{Formulation}
We can think of the policy as a test-time feature selection function $o(x, b): \mathcal{X} \times \mathbb{R} \mapsto \mathbb{B}^F$, where $\mathbb{B} \in \{0, 1\}$, and $b \in [0, 1]$ is given and specifies the fractional selection \emph{budget}.
Applied to $x$, $o(x, b)$ gives the binary selection vector $\mathbf{o}$ which splits $\mathbf{x}$ it into observed and unobserved parts such that $\mathbf{x}^m = [\mathbf{x}^\text{o}, \mathbf{x}^\text{u}]$.
$X^c$ denotes the fully observed $N \times F$ training set.
We assume that we have only sequential access to the missing-feature test instances $\mathbf{x}^m$.

\PM{Mean imputation}
A basic strategy is \textbf{mean imputation}: filling in with the mean value of the feature.
We simply replace the missing value in $X^m$ with their row mean in $X^c$.
Because our data is always zero-mean, this amounts to simply replacing the value with $0$.
\begin{align}
\mathbf{x}_\pi = \left[ h_i(x) : \left\{ \begin{array}{rl}
 h_i(x) &\mbox{ if $h_i \in \mathcal{H}_{\pi(x)}$} \\
 \bar{\mathbf{h}}_i &\mbox{ otherwise}
\end{array} \right. \right]
\end{align}

\PM{Gaussian Imputation}
If we assume that $\mathbf{x}$ is distributed according to a multivariate Gaussian $\mathbf{x} \sim \mathcal{N}(\mathbf{0}, \Sigma)$, where $\Sigma$ is the sample covariance $X^T X$ and the data is standardized to have zero mean, then it is possible to do \textbf{Gaussian imputation}.
Given a feature subset $\mathcal{H}_\pi$, we write:
\begin{equation}
\mathbf{x}_\pi = \begin{bmatrix} \mathbf{x}^\text{o}\\  \mathbf{x}^\text{u} \end{bmatrix} \sim \mathcal{N} \left( \mathbf{0}, \begin{bmatrix} \mathbf{A} & \mathbf{C}\\ \mathbf{C}^T & \mathbf{B} \end{bmatrix} \right)
\end{equation}
where $\mathbf{x}^\text{o}$ and $\mathbf{x}^\text{u}$ represent the respectively observed and unobserved parts of the full feature vector $\mathbf{x}$, $\mathbf{A}$ is the covariance matrix of (here and elsewhere, the part of $X^c$ corresponding to) $\mathbf{x}^\text{o}$, $\mathbf{B}$ is the covariance matrix of $\mathbf{x}^\text{u}$, and $C$ is the cross-variance matrix that has as many rows as the size of $\mathbf{x}^\text{o}$ and as many columns as the size of $\mathbf{x}^\text{u}$ \parencite{Roweis-gaussian-identities}.
In this case, the distribution over unobserved variables conditioned on the observed variables is given as
$\mathbf{x}^\text{u} \mid \mathbf{x}^\text{o} \sim \mathcal{N} \left( \mathbf{C}^T \mathbf{A}^{-1} \mathbf{x}^\text{o},\, \mathbf{B} - \mathbf{C}^T \mathbf{A}^{-1} \mathbf{C} \right)$.

\PM{Complexity}
After computing the complete covariance matrix $\Sigma$, which takes $O(N^3)$ time, we need to make $N'$ test-time predictions.
In the course of the predictions, we may need to compute at most $\min(N', 2^F)$ unique matrix inversions (again in $O(N^3)$).
The size of the matrices being inverted is proportional to the budget $b$, making this method slower for larger budgets.

\PM{k-NN Imputation and Prediction}
Instead of assuming anything, we could go directly to the source of the covariances---the actual feature values for all points in the training set.
The family of Nearest Neighbor methods takes this approach.
The algorithm for imputation is simple: find the nearest neighbors in the observed dimensions, and use their averaged values to fill in the unobserved dimensions.
For $\mathbf{X}^m$, we find the $k$ nearest neighbors with the highest dot product similarity $\mathbf{x}^{cT} \mathbf{x^m}$ or lowest Euclidean distance $\| \mathbf{x}^{c} - \mathbf{x}^{m} \|$, using only the features that are observed in $\mathbf{x}^{m}$.
For \textbf{imputation}, the unobserved values are set to the average across these nearest neighbors for that dimension.
Similarly, we do \textbf{classification} by returning the mode label of the nearest neighbors.

\PM{Complexity}
Finding the nearest neighbors by dot product similarity is $O(NF'^2)$, and Euclidean distance is the same with an additional constant term.
$F'S$ is the number of observed dimensions, which grows proportionally with budget $b$, making this method more expensive with increased budget.

\subsubsection{Learning more than one classifier}

\begin{figure}[ht]
\centering
\includegraphics[width=\linewidth]{../../figures/mdp_masks.pdf}
\caption[
Visualizing the discretization of the state space by the possible feature subsets.]{
The action space $\mathcal{A}$ of the MDP is the the set of features $\mathcal{H}$, represented by the $\phi$ boxes.
The primary discretization of the state space can be visualized by the possible feature subsets (larger boxes); selected features are colored in the diagram.
The feature selection policy $\pi$ induces a distribution over feature subsets, for a dataset, which is represented by the shading of the larger boxes.
Not all states are reachable for a given budget $\mathcal{B}$.
In the figure, we show three ``budget cuts'' of the state space.
\label{fig:mdp_masks}
}
\end{figure}


\PM{Subset clustering}
As illustrated in \autoref{fig:mdp_masks}, the policy $\pi$ selects some feature subsets more frequently than others.
Instead of learning only one classifier $g$ that must be robust to all observed feature subsets, we can learn several classifiers, one for each of the most frequent subsets.
This is done by maintaining a distribution over encountered feature subsets during training.

We use hierarchical agglomerative clustering, growing clusters bottom-up from the observed masks.
In the case of training $K$ classifiers, we need to find $K$ clusters such that the masks are distributed as evenly as possible into the clusters.
The distance measure for the binary masks is the Hamming distance; standard K-Means clustering technique is not applicable to this distance measure.
Each one of $K$ classifiers is trained with the \textsc{Liblinear} implementation of logistic regression, with $L_2$ regularization parameter K-fold cross-validated at each iteration.
