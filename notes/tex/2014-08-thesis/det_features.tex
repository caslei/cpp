%!TEX root=paper/paper.tex
\subsection{Features of the state}\label{sec:det_features}

\PM{State}
We refer to the information available to the decision process as the \emph{state} $s$.
The state includes the current estimate of the distribution over class presence variables $P(\mathbf{C}) = \{P(C_0), \ldots, P(C_K)\}$, where we write $P(C_k)$ to mean $P(C_k=1)$ (class $k$ is present in the image).
Additionally, the state records that an action $a_i$ has been taken by adding it to the initially empty set $\mathcal{O}$ and recording the resulting observations $o_i$.
We refer to the current set of observations as $\mathbf{o} = \{o_i | a_i \in \mathcal{O}\}$.
The state also keeps track of the time into episode $t$, and the setup and deadline times $T_s,T_d$.

\PM{Open vs Closed Loop}
An \emph{open-loop} policy, such as the common classifier cascade \parencite{Viola-IJCV-2004}, takes actions in a sequence that does not depend on observations received from previous actions.
In contrast, as presented in \autoref{fig:pomdp_summary}, our goal is to learn a dynamic, or \emph{closed-loop}, policy, which would exploit the signal in scene and inter-object context for a maximally efficient path through the actions.
Recall from \autoref{eq:policy} that our policy is determined by a linear function of the features of the state.

\PM{Dynamic features}
Since we want to be able to learn a dynamic policy, the observations $\mathbf{o}$ that are part of the state $s$ should play a role in determining the value of a potential action.
We include the following quantities as features $\phi(s,a)$:

\begin{tabularx}{0.8\linewidth}{p{0.23\linewidth}p{0.69\linewidth}}
$P(C_a)$ & The prior probability of the class that corresponds to the detector of action $a$ (omitted for the scene-context action).\\
$P(C_0|\mathbf{o}) \ldots P(C_K|\mathbf{o})$ & The probabilities for all classes, conditioned on the current set of observations.\\
$H(C_0|\mathbf{o}) \ldots H(C_K|\mathbf{o})$ & The entropies for all classes, conditioned on the current set of observations. \\
\end{tabularx}

Additionally, we include the mean and maximum of $[H(C_0|\mathbf{o}) \ldots H(C_K|\mathbf{o})]$, and $4$ time features that represent the times until start and deadline, for a total of $F = 1+2K+6$ features.

\PM{Augmented MDP}
Our system as any system of interesting complexity, runs into two related limitations of MDPs: the state has to be functionally approximated instead of exhaustively enumerated; and some aspects of the state are not observed, making the problem a Partially Observed MDP (POMDP), for which exact solution methods are intractable for all but rather small problems \parencite{Roy2002}.
Our initial solution to the problem of partial observability is to include features corresponding to our level of uncertainty into the feature representation, as in the technique of \emph{augmented} MDPs \parencite{Kwok2004}.

\subsubsection{Updating with observations}\label{sec:det_features_updating}

\PM{Class correlations}
The bulk of our feature representation is formed by probability of individual class occurrence, conditioned on the observations so far: $P(C_0|\mathbf{o}) \ldots P(C_K|\mathbf{o})$.
This allows the action-value function to learn correlations between presence of different classes, and so the policy can look for the most probable classes given the observations.
However, higher-order co-occurrences are not well represented in this form.
Additionally, updating $P(C_i|\mathbf{o})$ presents choices regarding independence assumptions between the classes.

\PM{Update methods}
We evaluate two approaches for updating probabilities: \emph{direct} and \emph{MRF}.
In the \emph{direct} method, $P(C_i|\mathbf{o}) = score(C_i)$ if $\mathbf{o}$ includes the observations for class $C_i$ and $P(C_i|\mathbf{o}) = P(C_i)$ otherwise.
This means that an observation of class $i$ does not directly influence the estimated probability of any class but $C_i$.
The \emph{MRF} approach employs a pairwise fully-connected Markov Random Field (MRF), as shown in \autoref{fig:det_figure1}, with the observation nodes set to $score(C_i)$ appropriately, or considered unobserved.

\PM{Learning MRF}
The graphical model structure is set as fully-connected, but some classes almost never co-occurr in our dataset.
Accordingly, the edge weights are learned with $L_1$ regularization, which obtains a sparse structure \parencite{Lee2006}.
All parameters of the model are trained on fully-observed data, and Loopy Belief Propagation inference is implemented with an open-source graphical model package \parencite{Jaimovich2010}.

\PM{Details}
An implementation detail: $score(C_i)$ for $a_{{det}_i}$ is obtained by training a probabilistic classifier on the list of detections, featurized by the top few confidence scores and the total number of detections.
Similarly, $score(C_i)$ for $a_{gist}$ is obtained by training probabilistic classifiers on the GIST feature, for all classes.
